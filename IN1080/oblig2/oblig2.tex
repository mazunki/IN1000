\documentclass{../../myassignment}

\courselabel{IN1080}
\exercisesheet{Assignment 1}{Measuring Voltages}
\student{Rolf Vidar Hoksaas}

\usepackage{xfrac}

\newcommand{\ohm}{$\Omega$ }
\newcommand{\volt}{$V$ }
\newcommand{\amperes}{$A$ }
\newcommand{\percent}{$\%$}
\newcommand{\micro}{$\mu$}
\newcommand{\kilo}{$k$}

\begin{document}

	\begin{problem}
		Calculate the required parasitic resistance $R_W$ in one inductor (corresponding to the wire resistance between $A_1$ and $A_1'$ in Fig. 3, so that 2 inductors in series draws 1.2A from a single 5V voltage source.

	\end{problem}

	\begin{answer}
		The resistance over one inductor will be half of the resistance that we find between both of the ends, as these are connected in series:
		\begin{eqnarray*}
			R_w &=& \frac{V_s}{2I_t} = \\
			 &=& \frac{5[V]}{2\cdot1.2[A]} = \frac{5[V]}{2.4[A]} = 2.08\overline{3} \Omega
		\end{eqnarray*}

	\end{answer}

	\begin{problem}
		Calculate the required length of copper wire in one inductor to get the resistance found in question 1 for a 0.4mm diam (AWG26) copper wire when the resistivity of copper is $\rho = 1.7\cdot10^{-8}$ \ohm m.
	\end{problem}

	\begin{answer}
		The total length required to find the aforementioned resistance is found by the formula $L=R_w \cdot \sfrac{A}{\rho}$, where A is the surface area where magnetism is inducted:
		\begin{eqnarray*}
			L &=& R_w \cdot \frac{A}{\rho} = R_w \cdot \frac{\pi r^2}{\rho} = \\ 
			&=& R_w \cdot \frac{\pi \sfrac{D}{2}^2}{\rho} = 2.08\overline{3}[\Omega] \cdot \frac{0.0002^2\pi [m^2]}{1.7\cdot10^{-8}[\Omega m]} = \\
			&=& 2.08\overline{3} \cdot \frac{4\cdot10^{-8}\pi}{1.7\cdot10^{-8}} = 15.4 [m]
		\end{eqnarray*}

	\end{answer}

	\newpage
	\begin{problem}
		Use the drill and fill one reel with copper wire, as shown in Fig. 4.  Grind the lacquer insulation off the ends (Fig.5 left).  Measure the resistance $R_W$ of your new inductor with RS-12.  Assume that the length of the wire in the coil is 15m.  Does this resistance correspond well with the value calculated inquestion 2?  You can not expect a perfect match, but make sure that the value is greater than about 2 \ohm; otherwise the current and the heat will be too high.
	\end{problem}

	\begin{answer}
		The resulted value is indeed greater, but close to 2\ohm. If the resistance were too low we could melt the copper, but on the other hand we don't want it too high as we actually want some induction to cause the stepper motor to move thanks to the magnetic force caused by the current. If it is too low, we can add more cable, but if it is too high we need to cut some copper instead.

	\end{answer}

\end{document}