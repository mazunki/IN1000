\documentclass[11pt]{article}
\usepackage{geometry} 
\geometry{left=1.5in,right=1.5in,top=1.5in,bottom=1in}
\renewcommand{\baselinestretch}{1.3}

\author{Mazunki Hoksaas}
\date{2022-10-05}
\title{Epistemology \\[1ex]\large Foundationalism vs. Coherentism}

\begin{document}
\maketitle
	\paragraph{} Epistemology is a subfield of philosophy, which seeks to answer questions regarding the idea of knowledge, by itself. Mainly, the field revolves around explaining what knowledge is, how we can obtain it, and how we can be certain the knowledge we have is true.

	\paragraph{} One of the most well-known, and generally agreed-upon, idea of what knowledge is and which properties it has is the theory of the Justified True Belief, originally described by Plato, analysis.

	While the name of the theory explains itself, we can clarify the meaning in parts: —For something to be a belief, an entity must consider the idea in their mind as truthful. —Knowledge must also actually be true for it to be valid under this school of thought. —Lastly, the subject must have a reasoning justifying their thoughts.

	\paragraph{} This set of rules seems reasonable, at first glance, and we can probably consider it sufficient... with some caveats. 

	First, how can we ever reach a point where we know something actually is true? That is, how can we distinguish whether our set of truths about the universe matches reality, and is not misguided by assumptions and mistakes? This is a hard point to answer, but is the main point this essay will focus on.

	Another thing which is not explained by the JTB analysis is what makes someone's justification of their beliefs enough. The Gettier problems and Goldman's causal theory go deeper into this problem. In summary, their criticism introduces another variable regarding the justification and the truth: there must exist a relationship between someone's justification and the veracity of their belief.

	\pagebreak{}
	\section*{Two theories of though}

	\paragraph{} As mentioned above, asserting the truthiness of any belief is a difficult task. It is an opinionated topic, and there are multiple schools of thought regarding it.

	We can interpret the word belief as the relationship a person has towards their reality, granted by their own perspectives and interpretations. This relationship is personal, in the sense that each person holds their own beliefs, and need not correspond with the actual reality, as explained by Descartes in his Meditations.

	When people perceive the outside real world, we do so by the means of our senses, and thus gain experiences we can reason about (Locke, 1954). As noted much earlier by Descartes (1641), these experiences are not immune to misconceptions. Because of this problem, Descartes focused on absolute truths, where there is no doubt about anything, which was hard, granted Descartes didn't even trust his own perceptions to be accurate.

	Both Descartes and Locke understood logical reasoning, meaning they would both be able to draw inferences from different pieces of knowledge, and thus gain a third, more complex, piece of knowledge. If some knowledge isn't drawn from anything else, we will consider it basic.
	
	\paragraph{} Putting these elemental authors next to each other presents an interesting point of discussion: Can we actually know anything for certain when our perceptions may be misguiding us? What is required to take the leap from a (justifed true) belief to consider it knowledge?

	To answer this question, we must make a decision. What does it mean to know something? In other words, what does it mean to possess knowledge? One way we could approach this question is by giving a definition of knowledge, so let's define it as the collection of ideas under a belief system which allow us to reason and comprehend the world around us, and thus be able to make cognate decisions.

	This definition does not answer the question, but it lays ground for the base rules on which we can build different schools of thought. Having a purpose gives us some restrictions for what we desire from our definition, after all.

	\pagebreak{}
	\section*{Foundationalism}
	\paragraph{} Understanding knowledge as a set of ideas which we hold to be true is useful, and foundationalists will leave no assumption unhandled. In the previous section, we defined basic beliefs, which is the quintessencial element of this train of thought. Beliefs have to be clear, distinct, and proven rationally for them to be valid.

	Although basic beliefs are the core of this branch, that does not mean all knowledge we can arrive at must be basic. In fact, most knowledge under this system is justified by inference. We can justify one unknown fact by another known fact, and thus grow ourselves a large set of known ideas.

	\paragraph{} While this looks all dandy, there are some problematic situations which we must avoid. If knowledge can be justified by other pieces of knowledge, we could potentially create dependency cycles in our argumentative graph.

	Argumentative cycles are disregarded as invalid under foundationalism, because there is no way to enter the graph from outside. In other words, a cycle doesn't come from a basic true belief.

	\paragraph{} Furthermore, an argumentative string would also be considered invalid if it's impossible to reach a basic belief by the end of it. In other words, we can say that under foundationalism, we must always be able to point at a basic belief (or multiple) from which we drew a conclusion for our more complex beliefs.

	\paragraph{} Last and trivially, foundationalism does not make any sense as the groundwork for truth if the basic beliefs we're using are not themselves justified.

	\subsection*{Justifying basic beliefs}
	\paragraph{} We have defined basic beliefs, but we still haven't shown how to justify basic beliefs. Given our rules for justifying more complex knowledge, this is necessary to gain any knowledge whatsoever.
	
	\pagebreak
	\paragraph{} BonJour (1988) claims it is impossible for any person to justify basic beliefs, since (1) the belief needs to have an explanation as to why it's true (2) the person needs to be aware of this explanation. It's impossible to hold this explanation prior to actually absorbing the idea empirically. 

	From this, it follow that the only way we can learn truths is to start with \textit{some} empirical belief... which goes against our rules: Basic beliefs must be justified epistemologoically.

	\section*{Coherentism}
	\paragraph{} While foundationalism, often considered an evolution of the Descartian method, seems like a very stable ground for any given set of knowledge... we have explained how it's complicated, at best, to actually gain any knowledge under the system. 
	
	Given the impracticality of foundationalism, and our need for actually having a system which can serve us outside of an ideal epistemological analysis, we need something else.

	Instead of relying on basic beliefs, we could instead rely on a network of likely beliefs (Quine, 1970). When we refer to a belief that is likely, we're simply saying we believe it to be true because our intuition and experiences have led us to have the idea that the thing is so. It seems to be true, so why not assume it is the case?

	Of course, assuming beliefs to be true, nilly-willy, doesn't seem like an assuring justification. In order to fit a belief into this system, it must at the very least be compatible with the other beliefs which we have granted for certain already. Furthermore, given multiple potential cases for a scenario to have occured, we should rely on the most simply and intuitive cause.
	
	In other words, what we are saying here, is that beliefs are allowed to be justified a posteriori, as long as it's coherent with what we already know.

	\paragraph{} There are some problems with this, though. We can see how knowledge potentially gains inertia in certain directions as knowledge expands. We haven't shown that the bias we're adding to the system is actually correct... which is not ideal. As an example, we could have chosen the most intuitive and simple explanation as the belief we introduce into the network when the correct explanation was never even consider in the list of candidates.
	
	Furthermore, we haven't enforced any relationship between beliefs at this point. Can a belief really justify another belief if there's no logical entailment between them? Since that doesn't seem very rational, we will require the justification to involve argumentative links between the nodes in the network.


	\section*{Compromise}
	\paragraph{} After looking at two fundamentally different approaches to the same problem, it seems to be the case that no matter which orientation we pick we will offend one group of people or another. Some people might be fundamentally against the idea of assuming something based on intuition and probability, while others might think development of ideas should happen in a less bureaucratic manner.

	\paragraph{} To say which group of thinkers is right might be impossible, and might even be a question without any real answer. Thomas Kuhn, although not exactly an epistemological philosopher, presents the idea of paradigms and revolutions of knowledge. His view on knowledge assumes we will eventually reach a dead end because of the path we've taken, and the tools we've chosen to use on a social level.

	We can draw an overlap between cohesionism and scientific knowledge, since much of the scientific research is done through experimentation, observation and assumptions. It feels important to note that even though we do our best, we will eventually err somewhere along our reasoning, and therefore coherentism will need to swap out certain components in its system, and we might even need to revise the whole network from scratch. This idea seems exactly what Kuhn describes in his book on Scientific Revolutions (1962).
	
	\pagebreak{}
	\section*{Pragmatism}
	\paragraph{} In this essay we have discussed what knowledge is, how we can obtain it, and how we can verify our achievements regarding knowledge.

	As an ending note, I'd like to ask whether it's actually important to know and share the truth, and promote truth over all else. Can we live without knowing the full truth regarding the universe and its things? Can we teach without being certain about things? If we see someone do something wrong (that is, incoherent to our own network of ideas), or say something we have justified true beliefs which would prove them wrong... how should we act? Is it right to tell them they're wrong?
	
	\paragraph{} In some contexts, knowledge is the fuel for thought and action. In other places, it's just a thing we have to deal with. This seems to be forgotten at times.

	\section*{References}
	\begin{itemize}
	\item[---] Descartes, René (1641) Meditations: http://www.classicallibrary.org/descartes/meditations/5.htm

	\item[---] Kuhn, Thomas (1962) The Structure of Scientific Revolutions. ISBN: 9780226458113

	\item[---] Locke, John (1956) Essays on the Law of Nature: https://www.britannica.com/biography/John-Locke

	\item[---] Plato: The Analysis of Knowledge: https://plato.stanford.edu/entries/knowledge-analysis/

	\item[---] Quine, W.V.O. (1970) The Web of Belief. 2nd edition. ISBN: 9780075536093


	\end{itemize}


\end{document}
