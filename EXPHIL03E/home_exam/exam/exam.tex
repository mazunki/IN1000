\documentclass{article}
%\usepackage{times}
\usepackage[margin=1.25in]{geometry}
%\renewcommand{\baselinestretch}{1.25}

\begin{document}
	\begin{flushright}
		University of Oslo
		\\\textbf{Course code:} EXPHIL03E
		\\\textbf{Candidate number:} 8025
		\\\textbf{Date:} 3rd December 2019
	\end{flushright}

\section*{Exam questions}
	

\subsection*{First question --- First option}
	    What is the difference between having sense perception of an object, for example, a piece of wax, and having knowledge about wax? Discuss with reference to two texts from the syllabus.

	\begin{flushright}
		\textbf{Word count:} 512
	\end{flushright}

	This questions asks about the difference between how we gain knowledge. There are mainly two ways in which this is gained, and one is rationally, and the other is empirically. I've chosen Descartes for the first author, as he explicitly discusses a piece of wax during his wax argument, and Plato as the second author, as he distinguishes between gained knowledge and innate knowledge. 
	
	% Descartes
	\subsubsection*{Descartes --- Wax argument}
		The wax argument by Descartes discusses how we actually don't know anything about a piece of wax which we thought we knew quite well. He starts out by trying to define it by its shape, its attributes and color, but finds out, as it melts, that it changes over time; so he learns that what he thought he knew about it was actually wrong. In consequence, he knew nothing about it, even though he thought he did. Nevertheless, he says that the real definition of the wax is defined by the things which we do percieve. 

		We know a specific piece of wax exists, though, because we can all observe it, and therefore is not only a part of our imagination. Universally, the idea about a wax itself must be true for a specific wax aswell, or the knowledge we have about wax in general must be wrong, as it would only be true for a smaller set of wax.

		From this we can say that by pure perception, Descartes would argue that you cannot actually know anything from a given thing. While as soon as he realised this he started disbelieving everything, methodically. He started from the bottom, and built his knowledge from rational thinking. He came down to an absolute truth: he couldn't disprove that he is a thinking ``thing''. 
		
		With this, and by proving the existance of the physical world, by again proving the existence of an all-good god which wouldn't trick us into believeing a fake material world. The reason god must be real is because he has an idea about this, and he knows such an idea cannot have come from the perceptional senses. By these ground premises he was able to rebuild his beliefs about reality thoroughly.

	\vspace*{1in}
	% Plato
	\subsubsection*{Plato --- Ideas}
		Starting with Socrates and the discussion he had with Meno ---as Plato brought forward the ideas of Socrates---, he was able to show him that a slave is as capable as any other to know things which they haven't learned explicitly. Socrates actually doesn't believe learning is possible, by itself, but teaching is. He believed the soul is immortal, and is the one that contains the knowledge, while the body merely remembers it.
		
		This way of teaching, dialectica, sets the base for many of the ideas of Plato. He uses the allegory of the cave to explain the difference between opinions and real knowledge. We all have different perceptions of reality, malfigured thoughts on how reality is defined. By climbing the ladder of knowledge, through rational thinking and by comparing different opinions, true facts about the world can be known. 

		Epistemologically, he explains a difference between of types of knowledge. On one side, a realm of ideas, of true facts that are absolute and universal truths which explain everything which has been expanded into the physical world. On the other side, we have not-so-perfect interpretations of these ideas, through the perceptions we have gained over time of the physical shapes ---which again have their own real shape in the realm of ideas---. 

\vspace*{1in}

\subsection*{Second question --- Second option}
	Is happiness relevant for finding out how we ought to act? Discuss with reference to two texts from the syllabus.
	\begin{flushright}
		\textbf{Word count:} 666
	\end{flushright}
	
	This question asks about how hapiness should affect our morality. It is hard to answer this without resorting to two main ideas: deontology, and utilitarism. I've chosen Kant's text on Morality for the deontological part, and JJC Smart's text on utilitarian ethics for the consequantialist part.
	
	% Deontology
	\subsubsection*{Kant --- Morality}
		Immanuel Kant was known for bringing empiricism and rationalism together into one single line of categorical imperatives. Nevertheless, he also discussed morality broadly. In his philosophical text about morality he talks about duties, motives, and categorical imperatives. 

		He says, since we are humans, we have have a moral duty to do good. Not because we are compensated for it, but because we should have a good will. This is the only morally correct thing to do. He discusses that actions which bring us pleasure are not good by themselves, but rather compensate us for the actions, so these do not count nor matter for the sake of our morality. 

		Kant also claims that since we cannot know the future, we don't know whether our actions will have a good or bad outcome. Therefore, we can only take a decision based on our intentions, which should be justified by a respect of the moral law. 

		Lastly, a categorical imperative is a motive for an action. He defines some rules which we should act upon, but argues that we should all be able to rationally come up with these ourselves, as we are all subjects with own capabilities. Firstly, he says that we should only do things that we all could do, without a wrongdoing. This means that it is never acceptable to lie, for instance, even though the scenario would, by other authors, require it. Secondly, he says that we must consider other people, above all else, as subjects, and not as tools. This doesn't mean that we can't get help from others, as by asking for help we respect and acknowkledge their capabilities, though. Lastly, we must acknowledge that we are part of a greater Kingdom of Ends, by which we are mere subjects.
		
		In conclusion, he says that our own happiness does not matter, but the happiness of others does matter. By this, we should all be happy. 

	% Utilitarianism
	\subsubsection*{JJC Smart --- Utilitarianism}
	Based on previous theories of Jeremy Bentham regarding utilitarianism, where the main premise is to gain the maximum good for the maximum number, Smart discusses different theories of this. It is hard to say that happiness is not a good thing, so utilitarianism is, by definition, relavant to how we should act. 

	Smart splits utilitarianism in two parts: act-utilitarianism, and rule-utilitarianism. They would each have their advantages and disadvantages, as they each have some problems, but are not inherently bad. An act-utilitarist would consider all the aspects of every single choice they take, meaning that they will always make the right choice, but has the disadvantage that it is very time consuming. A rule-utilitarist will take their choices based on a series of rules they have set on themselves. This means that they will not think through all the reasoning needed to make a perfectly right action, but will not strain themselves at all times.

	An important part to discuss is to distinguish between total and average happiness, as one could say that many people that are a little happy is better than a world where only one person is as happy as they can be. Or otherwise. We can also recognise and argue that that the goal should be to reduce sadness, in which case we are talking about negative utilitarianism.

	All in all, Smart is somewhere in the middle between utilitarianism and Kant's deontology. He understands there is no clear answer, as he proposes in a scenario where a man is innocent, but his death would result in the well-being of so many other people, preventing riots, and other innocent deaths.

	Personally, I believe Smart's point of view is really relevant for the future of artificial intelligence, as it's very hard to tell a machine what to do. It's morality should be measured numerically, and I think this is only possible through consequentialism, granting each action/consequence a value, and summing them up. These actions should be analysed from a deontological point of view, though, in my opinion.

\end{document}