\documentclass{article}

\begin{document}
	\section{}
	\subsection{Biology}
		CRISPR babies $\rightarrow$ Altering genomes to cure VIH \\
		Enhancing people might improve the standard, but puts average people at a lower status. \\
		Aristocracy of (pseudo)cyborgs?
		
	\subsection{Philosophy}
		Relationship philosophy $\equiv$ science. Love of knowledge. \\
		Philosophy: [Self, Society, God, Knowledge, Morality, Freedom] \\
		Answer these questions with pure science.
	
	\subsection{Cogito ergo sum}
		Proves the self. But... what does "I" mean? \\
		Intermingled bodies? Twins one head, but what is the correct? Should we force the selfhood to be mantained from what we believe it is? Or should we allow our understanding to change? \\
		Cloning = Two selves? Two bodies? Is it the same being?

	\subsection{Freedom}
		Do mutations restrict you from your freedom? 
		Freedom of choice vs laws of physics.

		If we are physical beings, we must follow physical laws. Is our reality of freedom just an illusion? \\
		Reaction of atoms is chemistry. Forms molecules. These join together in biology. 

	\subsection{Morality}
		Responsability of choices if there is no free will. Are people guilty if the laws of physics predetermine their actions?

	\section{Mayr --- Biology}
	\subsection{Vitalism}
		Vitalism: "Magic ingredient" living beings have. $\rightarrow$ Similar to phlogiston (magic ingrediant that makes things burn) \\
		Vitalism = "Lebenskraft", "Vis vitalis" \\
		Vitalism didn't fit other aspects of sciences, and other models make more sense. \\
		
	\subsection{Teleology}
		Study of purpose of things rather of the cause of them.

		Natural designs where they all have a role, so we can't be "clockworks". 
		Refutes Descartes' ideas of mechanics. 

		Nevertheless, has the same problem as vitalism

	\subsection{Darwin's evolution}
		Natural selection, benefits causing you to survive.
		
\end{document}
