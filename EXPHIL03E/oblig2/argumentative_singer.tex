\documentclass{myassignment}
\usepackage[spanish]{babel}
\usepackage[autostyle]{csquotes}
\MakeOuterQuote{"}
\newcommand{\q}[1]{``#1''}
\courselabel{EXPHIL03E}
\exercisesheet{Assignment 2}{Argumentative text}%%ignorewc

\begin{document}
	\begin{problem}
		Most human beings are speciesist, says Peter Singer (Singer 2 2017: 193). Discuss if Singer’s spesiecism argument includes not slaughtering animals and eating meat.%%ignorewc
	\end{problem}%

	\begin{answer}
		\section*{Peter Singer, a defender of all sentient life}%%ignorewc
		
		Peter Singer is an utilitarist philosopher who analyses a series of discussions which have been raised in the modern society, such as feminism topics, animal activist issues, and taking care of the environment, among others.

		\subsubsection*{Utilitiarism: definition}%%ignorewc
		
		Utilitarism was defined previously by authors like Jermey Bentham and John Stuart Mill as the concept of choosing the path that would bring \emph{the greatest amount of good for the greatest number}. Nevertheless, this definition usually only considered the greatest number of \emph{people}, and not the greatest number of \emph{sentient beings}, as Peter Singer suggests.

		Considering sentient beings instead of only people, or even human beings, is a revolutionary thought, and has inspired a lot of vegetarians and vegans. After all, it is an anthropocentric thought to believe only human beings or people should have an ethical value. There is no rational reason which explains why humans should matter more than non-human beings from a consequentialist point of view. We can therefore take this as a (renewed) starting point for our ethical ideologies.

		\subsubsection*{Non-antrophocentric philosophies}%%ignorewc
		
		In opposition to the previous ideas, we have pathocentric and biocentric views (ignoring the hollistic ideas), which, respectively, considers all sentient beings, and all living beings. Considering all sentient beings makes sense from a historical point of view, since not all human beings have been considered people throughout the years: Women and racialised people have not been considered subjects (as Kant puts it) since the beginning of time, but mere living beings that acts as means for men's purposes. This is still true in modern society for other animals, be it pets such as dogs and cats, or livestock animals, such as cows and pigs. 

		With this, we can draw a parallel between sexism, racism, and speciesism. If one of them is bad, the rest must by analogy also be bad, since there is no objective difference between one form of discrimination and another.

		Considering all biological life, nevertheless, is harder to argue for. For if not being able to feel pain, why would it matter how we treat them? From a purely deontological point of view, it would make sense to defend all forms of life, if our sense of what is good and correct were determined by \q{defending all forms of life}, but this is dogmatic from an utilitarian perspective. 

		On the other hand, since consequentalism tries to gain the highest benefit, it could be rational to defend all forms of life since this benefits us (all sentient beings, and their ecosystem) most.

		%TODO: Add stuff: act- vs rule-utilitianiarsm (JJC Smart)

		\subsubsection*{Consequences of pathocentric thinking}%%ignorewc

		By considering pathocentrism as our standpoint, it is obvious that all forms of pain should be diminished as much as possible, at least as an utilitiarist. This implies that torturing non-human animals and keeping them in poor conditions must be disallowed. 

		Some philosophers claim that it is not strictly necessary to make an effort to do good, as long as we don't do anything evil. This is a fallacy, as there is no direct implication from one to the other. If you don't do evil stuff, you're at least not evil, but if you have a moral obligation to help others: say the situation requires it, and you choose not to do anything; then you are not doing good. Kant already showed this difference between moral obligations and good actions. What he didn't consider, though, was the present state of things. This makes a difference, as even if the actions themselves are good, they might be inherently bad due to the circumstance in which they occur.

		Peter Singer has reflected on this, and says that animals like cows shouldn't be slaughtered if they are unhappy. Nevertheless, he considers that if a cow lives a happy life, then it doesn't make much of a difference whether they die for our benefit or not. After all, he says, the cow won't feel any pain after being dead, so we could eat a cow that lived happily without any moral rulebreaking.

		While having said that, he also mentions the production of cows being bad for the environment, and, since a higher demand requires a higher supply, the consequence is that eating/buying meat necessarily affects the greenhouse gas emissions. Consequently, he does not support the consumption of any animal on either of these bases.

		As a bonus, from a deontological pathological point of view, it would be considered objectification to produce animals just for our consumption.
 	\end{answer}
\end{document}