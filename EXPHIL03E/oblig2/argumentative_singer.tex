\documentclass{myassignment}
\courselabel{EXPHIL03E}
\exercisesheet{Assignment 2}{Argumentative text}%%ignorewc

\begin{document}
	\begin{problem}
		Most human beings are speciesist, says Peter Singer (Singer 2 2017: 193). Discuss if Singer’s spesiecism argument includes not slaughtering animals and eating meat.%%ignorewc
	\end{problem}%

	\begin{answer}
		\section*{Peter Singer, a defender of all sentient life}%%ignorewc
		
		Peter Singer is an utilitarist philosopher who analyses a series of discussions which have been raised in the modern society, such as feminism topics, animal activist issues, and taking care of the environment, among others.

		\subsubsection*{Utilitiarism: definition}%%ignorewc
		
		Utilitarism was defined previously by authors like Jermey Bentham and John Stuart Mill as the concept of choosing the path that would bring \emph{the greatest amount of good for the greatest number}. Nevertheless, this definition usually only considered the greatest number of \emph{people}, and not the greatest number of \emph{sentient beings}, as Peter Singer suggests.

		Considering sentient beings instead of only people, or even human beings, is a revolutionary thought, and has inspired a lot of vegetarians and vegans. After all, it is an anthropocentric thought to believe only human beings or people should have an ethical value. There is no rational reason which explains why humans should matter more than non-human beings from a consequentialist point of view. We can therefore take this as a (renewed) starting point for our ethical ideologies.

		\subsubsection*{Non-antrophocentric philosophies}%%ignorewc
		
		In opposition to the previous ideas, we have pathocentric and biocentric views (ignoring the hollistic ideas), which, respectively, considers all sentient beings, and all living beings. Considering all sentient beings makes sense from a historical point of view, since not all human beings have been considered people throughout the years: Women and racialised people have not been considered subjects (as Kant puts it) since the beginning of time, but mere living beings that acts as means for men's purposes. This is still true in modern society for other animals, be it pets such as dogs and cats, or livestock animals, such as cows and pigs. 

		With this, we can draw a parallel between sexism, racism, and speciesism. If one of them is bad, the rest must by analogy also be bad, since there is no objective difference between one form of discrimination and another.

		Considering all biological life, nevertheless, is harder to argue for. For if not being able to feel pain, why would it matter how we treat them? From a purely deontological point of view, it would make sense to defend all forms of life, if our sense of what is good and correct were determined by ``defending all forms of life'', but this is dogmatic from an utilitarian perspective. 

		On the other hand, since consequentalism tries to gain the highest benefit, it could be rational to defend all forms of life since this benefits us (all sentient beings, and their ecosystem) most.
 	\end{answer}
\end{document}