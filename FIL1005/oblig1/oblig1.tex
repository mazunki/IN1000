\documentclass{myassignment}

\exercisesheet{Oblig 1}{}
\courselabel{FIL1005}


\usepackage[UKenglish]{babel}
\usepackage[autostyle]{csquotes}
\MakeOuterQuote{"}
\newcommand{\q}[1]{``#1''}


\begin{document}

\textit{Et av de mest sentrale begrepene i Aristoteles’ metafysikk er begrepet substans.  Redegjør for de viktigste elementene i dette begrepet.}

Aristotles was born in Stagira, but resided most of his adult life in Athens, spending a great deal of his life studying at Plato's Academy \autocite{sep-aristotle}. While he wrote several works, some relevant ones for the topic at hand are Physics, Metaphysics and, mainly, Categories.

\section{Presocratics}
While his scriptures about Physics may not be an explanation to what substance is, in itself, it is still important to understand its relevance nonetheless. Before him, and before Plato, Presocratics philosophers were trying to explain the nature of things in an effort to pass from mythos to logos \autocite{sep-presocratics}. It is clear that the desire to have an understanding of things in general existed.

Traditionally, it was thought that the Gods were responsible for all change and matter in the universe \autocite{knopf2009}. It wasn't until the greeks from Miletus explored other cultures until they questioned this \textit{thought-to-be fact}. Thales declared water to be the first cause. Empedocles and Anaxagoras believed in pluralism: things aren't composed by one uniform and homogeneous thing, but rather as a mixture of all things, in miniscule parts. Democritus and Theophrastus explained that the different arrangements of \textit{particles} in its configurations and proportions made up the things as we know them.

While the presocratic philosophers mainly seeked what physical matter defined things, this idea wasn't enough for Aristotles. He needed an explanation for what makes things as they are, what causes them to exist, what is it which makes things such as they are. In his work Physics, he explains the principles which make objects change in nature. Change is an essential concept when understanding his philosophy.

\section{Ten categories}
According to Aristotles, all things could be described by a defined and limited set of categories, using these to form predicates about said thing \autocite{aristoteles-categories}. The list of categories includes substance, quantity, quality, relation, location, time, position, state, action, passion. The essence of a thing is defined by a set of predicates combining these, according to the philosopher.

Mainly, though, Aristoteles is interested in the substance of things. It is what is said about a thing in itself, and never of other things. It carries a meaning by itself, unlike the other categories, which rely on some referential point in order to carry meaning.

A small but important remark, is that he distinguishes two types of substances: namely primary and secondary substances. Primary substances are individual, unique and exclusive (apple, window, river, e.g), while secondary substances can be used to group together other things (animal, collection, family, e.g).

Unlike his predcessors, though, he doesn't believe these primary substances are in the things which can be described. We may say ``\textit{the apple in your hand is an apple}'', but it would be wrong to say the primary substance apple is in said apple.

\section{Actualization of potency}
As mentioned previously, Aristoteles talks about change in Physics. To understand change, one must first understand that not all substances can change into anything \autocite{aristoteles-physics}. You, Reader, can't be attributed the substance of Dog. You can, nevertheless, be attributed the substance of Human. It's impossible for you to change into a Dog, because the substance which you are will always remain unchanged.

One can change their size, though, which is a mutable trait, according to the potential room for change given by the substance. Every thing has potential to become different, while keeping the same predicate of substance, and thus being same thing. When change occurs, an actualization of this potential has happened. That is not to say the subject is a new subject, but only that it has matured.

\section{Four different causes}
To further understand how potential can turn into actualization, Aristoteles talks about causality. He lists four different \textit{whys}, which are meant to describe the reason for a thing to become altered, since random changes without no prior make no sense \autocite{sep-aristotle-causality}. To use the same example as before, where an individual human grows, we could describe its reason for change as such:

--- Efficient cause: Hormones interact with each other per the rules of biology and anatomy, leading to muscular and osseous expansion.

--- Material cause: The human body is capable of minor changes over time, provided enough nutrients are given.

--- Formal cause: Humans beings are creatures which grow over time.

--- Final cause: We must grow in order to become mature adults.


\printbibliography
\end{document}
