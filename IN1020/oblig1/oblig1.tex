Task 1
Identify a couple of values in this scenario. Can you imagine values that are more important to astudent than a teacher, and vice versa? Explain briefly, with examples. Also, name a couple of likely threat actors.

Task 2
CIT-services (confidentiality, integrity and accessibility) are essential features of information security. For each of them, consider the importance as well asthings that may go wrong. That is, what may represent a threat or danger to each security objective.

Task 3
Give two general security controls that in this case can help to achieve
	a) Confidentiality
	b) Integrity

Task 4
Explain the role of accountability and authentication. Would you recommend prioritizing these security goals (security services/properties) in our case? Justify your answer.

Task 5
The school Skolen has an overall authorization policy which includes the following:
	Pupils shall have access to view/read personal information about themselves, and only themselves.
	Teachers shall have access to read/view personal informationabout all pupils they teach in atleast one subject.
	Employees in the school administration shall have access toboth view/read and change allpersonal information about both teachers and pupils.
This policy applies regardless of how the information is requested, e.g. orally to the administration atthe school, or directly in one of the computer systems the school uses to store and process this type ofpersonal information. This means that the new digital learning environment must alsoenforcethesepolicies. Briefly explain the overall mechanisms/functions that must be in place in the computer system for this to be implemented.

Task 6
The school management also consider using a module in the system where pupils or their parentscan report absence and the reason for absence, and where teachers can register a pupil’s absence. Does the school have to pay special attention before they can use this modul? Justify your answer.

Task 7
Student Network: The school’s wireless network (WiFi) har been set up without a "password" for encryption. Why is this a bad idea? Does turning on "encryption" in the wireless network affectwhether it is safe to allow students and teachers to share a wireless network? Explain briefly.

Task 8
Think like a «hacker»: As a student, you will try to change a grade in the system. Give an example ofhow you would do this! (PS. No exact answer :-))