\documentclass{../../myassignment}

\usepackage{array}

\courselabel{IN1030}
\exercisesheet{Oblig 2}{Bruk og brukerundersøkelser}

\begin{document}
	\subsection*{Oppgave 1 --- Observasjon av bruk}
	\paragraph*{a)}

	% Planlegg datainnsamlingen. Beskriv følgende:
		% Hva du forventer å finne ut ved å studere interaksjonen mellom en bruker og den digitale løsningen, og hvem du tror faller innenfor målgruppen til systemet?
		Ved {\aa} studere interaksjonen mellom flere brukere av det eksisterende Vy-systemet og dens maskiner forventer jeg {\aa} finne ut hvordan folk generelt sett tiln{\ae}rmer seg en togstasjon. Jeg har allerede opplevd selv at problemer allerede kan oppst{\aa} f{\o}r vi engang n{\aa}r det digitale punktet. Eksempler inkluderer at vi ikke vet hvor stedet ligger, ikke finner inngangen, eller ikke kjenner igjen maskinene. 

		I noen omstendigheter vet ikke brukeren om protokollene heller, og det finnes ingen lett tilgjenglige instrukser. Forskjellige aldersgrupper og forskjellige bakgrunner vil ha forskjellige problemer, og m{\aa}ter {\aa} l{\o}se disse problemene p{\aa}. Jeg {\o}nsker {\aa} f{\aa} et innblikk i disse grupperingene, og finne frem generiske l{\o}sninger og betraktninger man m{\aa} tenke p{\aa}.

		% Opp til 5 relevante oppgaver du vil be deltakeren om å utføre i systemet.
		\begin{itemize}
		\item[---] Kj{\o}pe en billett via datamaskin. 
		\item[---] Kj{\o}pe en billett ved en vilk{\aa}rlig og ukjent stasjon. 
		\item[---] Refundere en billett.
		\item[---] Finne frem fra en stasjon til en annen stasjon, uten aa bruke eksterne hjelpemidler (alts{\aa} ved {\aa} bare bruke Vy-tilbudte systemer). 
		\item[---] F{\aa} kontakt med en ansatt for {\aa} sp{\o}rre etter hjelp (toalettet? spr{\aa}khjelp? assistanse? {\dots}?)
		\end{itemize}


		% Deltakeren. Er hun/han en del av målgruppener
		Deltakeren jeg har valgt er en del av den yngre m{\aa}lgruppen. Yngre mennesker er en del av fremtiden, og burde kanskje derfor bli prioritert n{\aa}r det gjelder innovasjon, men p{\aa} den andre siden s{\aa} er det gjerne de yngre gruppene som trenger minst hjelp. Det beste ville v{\ae}rt {\aa} sp{\o}rre flere ``typer'' mennesker, og f{\aa} et bredt innblikk av systemet. {\AA} ha muligheten til {\aa} gj{\o}re denne observasjonen i utlandet gir meg muligheten til {\aa} observere hvordan systemet fungerer her, for {\aa} s{\aa} sammenligne med systemet i Norge, og eventuelt se forskjellen mellom hvordan folk oppf{\o}rer seg annerledes her enn der.

		% Din relasjon til deltakeren. Hvordan tror du dette kan spille inn på dataen du samler inn?
		Siden jeg er ganske kjent med dev vedkomnmende deltageren ville det kanskje v{\ae}re lett {\aa} tenke seg at jeg vil forvrenge resultatene av observasjonen, men jeg er ganske vant med {\aa} v{\ae}re ekstern observat{\o}r uten {\aa} innblande meg. Desutten vil jeg informere om dette f{\o}r observasjonen, for {\aa} unng{\aa} innblanding.

		% Hvordan du vil registrere data under observasjonen, som for eksempel medpenn/papir, opptak av lyd, fotografier, video, annet?
		Jeg har tenkt {\aa} forklare gj{\o}rem{\aa}lene hver for seg, og filme diverse snutter som jeg videre analyserer hjemme. Jeg tror det vil v{\ae}re lettere {\aa} analysere detaljer i etterkant ved {\aa} kunne se ting om igjen, med god tid. Alternativet ville v{\ae}rt {\aa} skrive ting opp etterhvert som de skjer, eller bare bruke hukommelsen. Om deltageren ikke {\o}nsker {\aa} bli filmet vil jeg forbrede handligssekvens-tabellen i forkant, for {\aa} s{\aa} skrive fortl{\o}pende det jeg ser. 

		Konklusjonen og utdypelsen av analysen m{\aa} skje i etterkant uansett, men ikke alt for lenge etter, siden hukommelsen ogs{\aa} har et betydning i helheten.


	\newpage

	\paragraph*{b)}

	Personvern er en viktig del av integriteten til alle og enhver. {\AA} vite hvilken informasjon av en selv som blir brukt av andre er viktig for mange, og fra et filosofisk standspunkt ville det ver uetisk {\aa} bruke informasjon om andre uten deres modne, fullverdige, og frie sammtykelse. 

	Det er i tillegg p{\aa}krevd av alle land innenfor EU {\aa} f{\o}lge GDPR (General Data Protection Regulation), p{\aa}krevd siden 2018 med bakgrunn p{\aa} etiske prinsipper som tidligere ble brutt. {\AA} ikke f{\o}lge denne loven kan risikere selskapet store p{\aa}legg, og eventuelt bli lagt ned om ledelsen ikke tilrettelegger l{\o}sninger, og f{\o}lger disse.

	\paragraph*{c)}
	\begin{quote}
		Sammtykelse for student-observasjon

		Jeg, \_\_\_\_\_\_\_\_\_\_\_\_\_\_\_, godtar, forst{\aa}r, og samtykker med at mitt navn blir registrert i en del av unders{\o}kelsen som ble gjort den \_\_ februar i 2020. Denne informasjonen vil bli brukt til akademiske form{\aa}l som en del av en student-unders{\o}kelse, og vil ikke bli behandlet i en komersiell sammenheng.

		Ved {\aa} signerere dette dokumentet lar jeg Rolf Vidar Mazunki Hoksaas observere og ta opp det jeg gj{\o}r, for {\aa} videre analysere situasjonen, valgene, og omheng. Resultatene av observasjonen vil kunne bli publisert, enten med, eller uten mitt navn. Dette gj{\o}r jeg av min egen fri vilje, uten noen personlig gevinst, eller noen form for press. 

		\vspace{1cm}

		\_\_\_\_\_\_\_\_\_\_\_\_\_\_\_, Barcelona. \_\_ februar, 2020.
	\end{quote}


	\paragraph*{d)}
	% Gjennomfør en pilotundersøkelse av den planlagte datainnsamlingen. Rapporter kort hva du erfarte. Hvilke lærdommer tar du med deg til hovedundersøkelsen? Gjør eventuelle modifikasjoner på planen og beskriv disse.

	\paragraph*{e)}
	% Gjennomfør hovedundersøkelsen (observasjonen). Husk å ta et bilde for å dokumenteregjennomførelsen.

	\subsection*{Oppgave 2 --- Analyse: Sekvens av handlinger}
	\paragraph*{a)}
	En handlingsekvenstabell er en ``to-dimensjonell''-tabell som beskriver hvordan en bruker og en maskin kommuniserer. Langs ``y-aksen'' har vi tidsrommet, og langs ``x-aksen'' har vi i grunnen to spalter, som igjen er hver delt i to.

	Til venstre har vi brukeren, og til h{\o}yre har vi maskinen. Ytterlig til venstre fremkommer alt brukeren gj{\o}r som ikke maskinen ser, og tilsvarende ytterlig til h{\o}yre fremst{\aa}r det maskinen gj{\o}r i bakgrunnen. I de to gjenst{\aa}ende spaltene i midten ser vi hva brukeren og maskinen sier til hverandre.

	I de ytterlige kolonnene er mye informasjon, som muligvis er viktig {\aa} formidle til hverandre.

	Form{\aa}let med denne type tabell er \aa{} kunne analysere hvordan brukere interagerer med maskiner. Vi vil kunne se hvilke forst{\aa}else-problemer som finnes i systemet, for {\aa} dermed ha mulighet til {\aa} forbedre disse.

	\newpage

	\paragraph*{b)}  % tabell

	\begin{tabular}{ | >{\centering}p{10em} || >{\raggedleft}p{10em} | >{\raggedright}p{10em} || >{\centering\arraybackslash}p{10em} | }
	\hline
	\multicolumn{2}{|c|}{Human} & \multicolumn{2}{c|}{Machine} \\\hline
	Environment & {To Machine} & To Human & System design\\\hline\hline
	a & b & c & d
	\end{tabular}

	% more tablesssss

	\paragraph*{c)}  % analyse av tabell

	% Analyser tabellen du fylte ut i oppgave 2b. Hvordan oppfatter du interaksjonen mellombruker og system? På hvilke måter kan interaksjonen forbedres eller bli gjort annerledes?

	\subsection*{Oppgave 3 --- Øvelse: Oppmerksomhet og distraksjon}

	{\AA} skru av telefonen/nettet er noe jeg er ganske vant med. Jeg bruker telefonen, datamisknen, og b{\ae}rbaren aktivt nesten hver dag for tiden, men (nesten) aldri n{\aa}r jeg er p{\aa} jobb. Hvis noen ringer meg p{\aa} jobben s{\aa} svarer jeg bare dersom det er sjefen, eller andre viktige jobb-kontakter. Hvis ikke det er noen jeg har aktivt lagt til i unntakene til ``Do Not Disturb''-moduset vil jeg ikke f{\aa} notifkasjon f{\o}r etter jeg har tid til {\aa} sjekke meldingene. Jeg anbefaller alle {\aa} ta seg tid til {\aa} sette opp telefonen(e) sine for {\aa} bare f{\aa} de meldingene de er interessert i, og ikke alt annet tull. Tross alt, er det er din telefon, og ikke du som er telefonen sin. 

	{\AA} bruke telefonen som et verkt{\o}y er en positiv ting, s{\aa} lenge man ikke er avhengig av denne. Skru gjerne av telefonen i flere uker om du f{\o}ler den tar kontroll over livet/hverdagen din, siden det betyr at det har g{\aa}tt for langt. Det er fullt mulig {\aa} leve uten telefon selv i dag, der alle andre forventer deg til {\aa} f{\o}lge med p{\aa} lasset. Det handler bare om {\aa} tilpasse seg, ha selvkontroll, og v{\ae}re bevisst om hva en driver med. 

	\subsection*{Oppgave 4 --- Spørsmål til pensum}

	\begin{itemize}
		\item[---] Har det v{\ae}rt noen endring gjennom historien relatert til hva folk foretrekker mellom ``objektive'' og ``ekspressive'' statistikker/m{\aa}linger?
		\item[---] Trenger vi {\aa} m{\aa}le ting objektivt for {\aa} komme til konklusjoner, eller er det nok {\aa} f{\o}lge ``bobler'' som kunstig intelligens kan vise oss?
		\item[---] Kan vi relatere forskjellige m{\aa}leteknikker til forskjellige typer intellgense? Vil noen mennesker ha det lettere for {\aa} forst{\aa} ting om vi presenterer informasjon p{\aa} en ``strippet'' og ``tall-orientert'' m{\aa}te enn andre? 
	\end{itemize}

	\subsection*{Oppgave 5 --- Refleksjon}

	% Skriv noen setninger om hva du har lært gjennom å jobbe med denne obligen. Hva likte du?Var det noe som var tankevekkende, engasjerende? Skriv minimum 3 setninger.

\end{document}